\section{Plasmon}
\label{sec:TheoPlas}

Before delving into nanoplasmonics, this section will provide an introduction to the concept of a plasmon
and elaborate important aspects relevant to this laboratory practice.

A plasmon is a quasiparticle such as a phonon. In contrast to a phonon, the particle is not a vibration of 
the lattice but a collective oscillation of the electrons in response to an interaction of a electromagnetic field, such as light. 

\subsection*{Bulk plasmons}

Graphically, the plasmon is a motion of electrons, bound to a lattice of positively charged atomic nucleus. When an electric field pushes the 
electrons aside they start wobbling around the nuclei with the plasma frequency $\omega_P$ till the oscillation is dampened. 
These collective oscillations of charge are known as plasmons, specifically referred to as bulk plasmons to differentiate them from surface plasmons discussed later. Since bulk plasmons are longitudinal waves, they are unable to interact with transverse electromagnetic fields and therefore cannot be excited or scattered directly by direct irradiation.

\subsection{Particle plasmons}

Particle plasmon can be seen are a special case of plasmons which are confined in three dimensions. This 
confinement in the nano scale introduces interesting effects as it links the macroscopic properties of the 
material to the quantum mechanical effects reoccurring at the nano scale. In the following we are going to 
investigate this complex effect for nanorods. 


