%Matteo Kumar - Leonard Schatt
% Fortgeschrittenes Physikalisches Praktikum

% 1. Kapitel Einleitung

\chapter{Introduction}
\label{chap:einleitung}

In the following, we are going to investigate the effects of nanoplasmonics. As nanostructuring is gaining importance in science and industry, for example in the context of computing or building organic solar cells, this presents a great opportunity to be introduced to this field. Nanoplasmonics is the study of the effects of plasmons, which are strongly confined to the nanoscale. This confinement connects the classical phenomena of plasmons to the world of quantum mechanics. This reappearance of the discretization of energy levels due to the confinement allows the observer to extract information from the sample well below the diffraction limit of $\frac{\lambda}{2}$. 
