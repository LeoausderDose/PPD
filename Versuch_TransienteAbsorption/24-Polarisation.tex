\section[Polarization effects]{Influence of polarization on transient absorption spectroscopy}
\label{sec:TheoPolarization}

This section explains why polarization effects are not important in this our setup.

In general, there are polarization effects. As LASER light is often polarized, polarization effects can occur. The effects happen because
the absorption is sensitive position of to the transition dipole moment to the electric fields. This leads to a photo selection in the absorption. When probing with a polarized probe beam 
we observe this photo selection in our data, depending on the angle we are penetrating the exited material. In this case this effect is not important as we are probing in the scale of nanometers, when the vibration of the molecule is in the range of
the molecule is in the range of picoseconds.

