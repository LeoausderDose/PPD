\section{Absorption, fluorescence and phosphorescence}

The following section gives a brief introduction to principles important for spectroscopy.

Spectroscopy is the study of the interaction between electromagnetic radiation and matter. By examining the interaction of radiation with matter, spectroscopy provides valuable information about the composition, structure, and properties of substances. The basic principle of spectroscopy involves the effects such as absorption, emission, scattering and many more. In our case, the electronic transitions, therefore, absorption and emission are going to be the dominant processes.

For pedagogical reasons, the effect of transitions is first introduced for singlet state transitions. In a single molecule, the quantum states $\ket{\Psi_{initial}}$ and $\ket{\Psi_{final}}$ are orthogonal. This leads to a transition dipole moment $M$ of 0 for the classical Hamiltonian. When introducing a perturbation in the case of a weak perturbation, the orthogonality of those two states is lifted. The electromagnetic perturbation in our case couples the two states and the transition dipole moment (TDM) are nonzero. It is important to note the TDI for emission and absorption are the same as $M$ is an observable and therefore a real number and
\begin{equation}
    \bra*{\Psi_{initial}}\hat\mu\ket*{\Psi_{final}} = M_{if} = M_{fi} = \bra*{\Psi_{final}}\hat\mu\ket*{\Psi_{initial}}
\end{equation}
is symmetric.

When absorption happens, a photons of a corresponding energy to the transition is absorbed. This leads to an 
absorption pattern that reflects the structure of the energy levels of an atom as depicted in 








\section{Transient absorption}
\label{sec:TheoTransAbs}

In the following section, the effect of transient absorption is explained