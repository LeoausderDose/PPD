\section{Influence of the measurement method}
\label{sec:Method}

In the following section the difference in results performing white light reflectometry in reflection
and transmission is analyzed and discussed.

As can be seen in Figure ref, the measured spectra in transmission and reflection show the same tendencies with regard to the position and number of minima and maxima. The films which are coated with a higher speed of rotation show less minnima and maxima in the 
observed spectral range. This is an indicator for a lower layer thickness as can be concluded from equation \ref{eq:lamdathick}, as we expect since faster spinning in coating leads to 
bigger centrifugal forces on the solution during coating. This intuitive explanation is also reflected in the experimentally determined Schubert equation. 


\begin{figure}[ht]
    \centering
    \begin{subfigure}[b]{0.70\textwidth}
        \centering
        \input{Bilder/Auswertung/VglMessmeth/VglMessmethrefl500adn5000.tex}
        \caption{$y=x$}
        \label{fig:y equals x}
    \end{subfigure}  
    

    \begin{subfigure}[b]{0.70\textwidth}
        \centering
        \input{Bilder/Auswertung/VglMessmeth/VglMessmethtrans500adn5000.tex}   
        \caption{$y=3\sin x$} 
        \label{fig:three sin x}
    \end{subfigure}

    \caption{Spectra measured in reflectance and transmission on a thin film of PS on Glass. The films are coated with different coating speeds.}
    \label{fig:three graphs}
\end{figure}