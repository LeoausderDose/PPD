\section{Interference}
\label{sec:refleclambda}
To observe constructive interference at the surface of a thin film, the initial beam and the twice reflected beam need to have the same phase. Therefore, the optical path length inside 
the medium between those two beams has to be of the wavelength $\lambda$ of the incoming light. The optical path length $l$ inside the film is given by 

\begin{equation}
    l = 2 n d
\end{equation}
with $n$ being the refractive index and $d$ being the thickness of the thin film. The factor 2 occurs as the beam has to pass the film twice to get to the initial surface. The phase jumps at the interfaces
can be ignored since they do not change the results in this case. But any integer multiple of a wavelength can cause constructive interface which makes the constraint on the wavelength 
\begin{equation}
    \lambda = \frac{2 n d }{m}  \label{eq:lamdathick}
\end{equation}
with $m$ as the diffraction order.
