\section{The q-space}
\label{sec:qSpace}

The q space is a representation of the real space in terms of frequencies. As the real space observed shows limitations, 
the q space has to reflect these limitations too. The limits are given by the bounds in real space, which translate to the 
Fourier space as follows:
\begin{equation}
    \frac{2\pi}{l_{\mathrm{max}}}<q < \frac{2\pi}{l_{\mathrm{min}}}.
    \label{eq:qspace}
\end{equation}
Here, the maximum dimension $l_{max}$ in our image is the length of the image, while the minimum resolution object is one pixel. 

To resolve the diffusion, also have to take into account the temporal limitations connected to our total time $T_{\mathrm{exp}}$ of the experiment. If the 
particle has not moved at least one pixel in this time, we can not detect a change as it is too slow. The same is true for the fast particles. If the particle
disappears from the image in the timescale of the shutter speed $T_{\mathrm{shutter}} $, it can not be detected.  

The displacement for a Brownian motion is given by 
\begin{equation}
    \Delta x^2 \propto  D_{\mathrm{m}} t   \Rightarrow \Delta x = \pm \sqrt{D_{\mathrm{m}} t}.
\end{equation}
Now we insert this relation into the minimal and maximal resolvable displacements as described in \cref{eq:qspace} and arrive at 
\begin{equation}
    \frac{1}{\sqrt{D_{\mathrm{m}} T_{\mathrm{exp}}}} < q < \frac{1}{\sqrt{D_{\mathrm{m}} T_{\mathrm{shutter}}}}
\end{equation}
as limitation to the representation of diffusion in our setup. 