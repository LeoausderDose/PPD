%Matteo Kumar - Leonard Schatt
% Fortgeschrittenes Physikalisches Praktikum

% 1. Kapitel Einleitung

\chapter{Introduction}\label{chap:einleitung}

Diffusion describes the motion of particles from an area of high concentration to an area of low concentration. It is based on the Brownian motion which is stocastical, thermal movement~\cite{Demtroeder.2021}. Diffusion is a process of incredible importance in many fields. It plays a role in engineering (e.g.~diffusion of powders while sintering~\cite{Johnson.1963}), in the semiconductor industry (e.g.~diffuson of excitons~\cite{Mikhnenko.2015}) and biological systems (e.g.~difusion of gases in the lung~\cite{Liptay.2009}). Therefore, understanding diffusion is essential for a broad spectrum of sciences. Consequently, many techniques have been developed for quantisizing this process. \\
In this practical course, we used differential dynamic microscopy (DDM) for measuring the diffusion constant of polystyrene beads. For imaging, the concept of lightsheet microscopy was introduced which allows planewise imaging of a flourescent sample. 
