\section[Alternative methods]{Alternative methods for measuring diffusion}
\label{sec:AlternativeMethods}

Light sheet microscopy / DDM is not the only method to determine the diffusivity. In the following comparable methods are introduces and briefly 
commented. 

\begin{itemize}
    \item \textbf{Single Particle Tracking} (SPT) is a microscopy tool that enables the tracking of individual fluorescently labeled particles 
    within a medium or living cells. By observing the trajectory of a single particle in the cell membrane, 
    information about its dynamic behavior over time can be obtained. In particular, the diffusion coefficient D can 
    be measured.
    \item \textbf{Dynamic Light Scattering} (DLS) is a powerful technique used to study the diffusion
    behavior of molecules in solution. It provides a mean to calculate the diffusion
    coefficient and, consequently, the hydrodynamic radii of the macromolecules. DLS works 
    by analyzing the time-dependent fluctuations in the scattered light caused by the motion
    of the macromolecules. By measuring these fluctuations, the rate of diffusion of the
    molecules through the solvent can be determined.
    \item \textbf{Fluorescence Recovery After Photobleaching} (FRAP) is a technique used in biophysics to study the movement of molecules e.g. within living cells. In FRAP, a group of fluorescently labeled molecules is photobleached
    within a specific region of interest. The fluorescence recovery that occurs is then
    monitored as the bleached molecules exchange with the surrounding unbleached molecules in
    that region. This allows for the examination of molecular diffusion dynamic.
\end{itemize}
  
These examples represent only a few of the many methods available for measuring diffusion. However, Differential Dynamic Microscopy (DDM) offers distinct advantages over other techniques. DDM stands out due to its simplicity in setup, the ability to remove static contributions along the optical path, the simultaneous utilization of different microscopy contrast mechanisms, and the flexibility to choose an analysis region like a scattering volume. In many cases, DDM outperforms tracking approaches and correlation techniques~\cite{Berne.2013}. Recently, Fourier-domain analysis of imaging data has emerged as a potential method for combining the benefits of microscopy experiments and scattering measurements while overcoming some of their limitations [9]. However, DDM has not yet become a routine characterization tool primarily due to its computational cost, complexity, and lack of algorithmic robustness.