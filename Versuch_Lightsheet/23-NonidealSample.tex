\section[Polydispersity and Flow]{Leaving the idealized world: Polydispersity and Flow}
\label{sec:PoldysAndFlow}

When we do experiments, we leave the ideal framework of theory. Instead, we enter the real world. For two unrealistic 
preconditions, we want to offer a solution on how to better fit the prediction to the real data.

The first aspect, that we want to take into account is the polydispersity of the 
used materials. As there can be aggregate formation in our solution, not all particles do have the
same size. This changes the diffusivity in the particles and can be corrected using an adjusted 
formula 
\begin{equation}
    |F_D(q,\Delta t)|^2 = A(q)[1- g(q, \Delta t)] +B(q).
\end{equation}

The easies way to correct the polydispersity by using a stretched exponential
\begin{equation}
    g(q,\Delta t) = \exp(-\left(\frac{\Delta t}{\tau}\right)^{\alpha (q)})
\end{equation}
with $\alpha (q) $ as single stretch parameter~\cite{Wulstein.2016}. Futhermore, a cumulant fit function could be used.

The second aspect that we can correct is having a directed flow in the sample. The problem of directional flow is that it breaks the isotropy. 
The first option would be to correct the motion by moving a cut out from the photo across the picture and analising this 
frame. The speed of the frame can be determined by analysing the centre of gravity of the permanently visible objects. 
An other idea would be to disconnect $g(\vec{q}, t)$ for diffusion and other system inherent processes to 
\begin{equation}
    g(\vec{q}, t) = g_{\mathrm{diffusion}} \cdot g_{\mathrm{system}} .
\end{equation}