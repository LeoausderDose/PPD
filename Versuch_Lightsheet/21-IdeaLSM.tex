\section{Differential dynamic microscopy}
\label{sec:DDM}

In the following section the general principle of differential dynamic microscopy (DDM) is explained. The concrete implementation of DDM
in the form of lightsheet microscopy is explained in (ref).

DDM is a special version of optical microscopy. The experimentalist uses tools of Fourier optics to extract information about the 
diffusivity. This is even possible below the diffraction limit. The displacement of the particles can be represented in the integral 
\begin{equation}
    \sigma^2(\Delta t) = \int |D(x,y,\Delta t)|^2 dx dy
\end{equation}
with D being the difference signal $D(x,y,\Delta t) = I(x,y,\Delta t) -I(x,y,\Delta t =0)$. In our case $I(x,y,\Delta t)$ is the intensity i our our 
picture with the timestamp $\Delta t $ compared to the first picture with $I(x,y,\Delta t = 0)$ \cite{Cerbino.2008}. Parcevals theorem tells us that all information
contained in the special data has to be contained in the Fourier modes as 
\begin{equation}
    \int |D(x,y,\Delta t)|^2 dx dy = \sigma^2(\Delta t) = \int |F_D(u_x,u_y,\Delta t)|^2 du_x du_y
\end{equation}
with $F_D$ being the Fourier transform of D \cite{Butz.2012}. As the sample is isotropic a in the sample and across time, the 2D spectrum 
can be turned in the $(u_x, u_y)$ plane and therefore all relevant information can be transferred to a 1D spectrum of $u = \sqrt{u_x^2+u_y^2}$. To make data comparible to 
"normal" scattering data, the wave vector $q = 2\pi u$ can be calculated. 

The data in the Fourier space can be well described using the formula
\begin{equation}
    |F_D(q,\Delta t)|^2 = A(q)[1- \exp(-\frac{\Delta t}{\tau (q)})] +B(q)
\end{equation}
assuming Brownian motion for the particles~\cite{Berne.2013}. The equation contains of two main terms. The $B(q)$ term reprents the 
noise e.g. from the detector. The other term represents the diffusion process, with $A(q)$ as scaling factor related to quantities such as concentration, 
interaction strength with the light and incoming intensity. It is empirically generated from the data by fitting. $\tau$ is 
\begin{equation}
    \tau = \frac{1}{D_m \cdot q^2}
\end{equation} 
with the diffusivity $D_m$ for a Brownian motion. The total process of DDM is represented in \cref{fig:DataTreatDDM}.


\begin{figure}[ht]
    \centering
    \includegraphics[width = \textwidth]{Bilder/Theory/SchematicDDMEdit.pdf}
    \caption{Schematic representation of the DMM data treatment process. Edited from~\cite{Struntz.2017}}
    \label{fig:DataTreatDDM}
\end{figure}