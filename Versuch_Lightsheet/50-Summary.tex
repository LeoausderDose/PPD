%Matteo Kumar - Leonard Schatt
% Fortgeschrittenes Physikalisches Praktikum

% 5. Kapitel Fazit
\chapter{Summary}\label{chap:fazit}
After working with the confocal microscope in the Bachelor's practical course, we now had first hand on experinence with a similar imaging method, the lightsheet microscope. We experienced the latter as a fast and high contrast method for 3D imaging of biological probes. Hereby, we found the thickness of the lightsheet to be a crutial parameter for the quality of the imaging. Further, we used DDM to determine the diffusion coefficient of polystyrene beads in an agarose solution. DDM turned out to be an elegant method, just requiring time resolved plane images instead of e.g.~elaborate particle tracking. 

