\section{Identification of metal foils by their absorption edges}
\label{sec:absorb}

In the following section, we are going to identify an unknown metal foil due to its absorption edges. The used setup is described in section (ref).
The setup is able to measure the transmitted intensity in regards to a diffraction angle. The measured angle $\omega$ (=$2\Theta$) can be varied from 
5.0° $\leqslant \, \omega \, \leqslant$ 51.0°. In the beam path, we placed two foils of thickness $d = \SI{2.5e-3}{cm}$ with a number $N$ on them. In the following, they shall be called \textit{F11} and \textit{F8} for foil 11 and 8.
The thickness $d$ is treated as if it had no deviation. First, as in spectroscopy, a reference measurement is performed. In the 
second step, the foils are put in place and the measurement is repeated. 

