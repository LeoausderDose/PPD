
\newpage

\section{Crystal structure of Zircon/Zirconium(IV)silicate (ZrSiO4)}
\label{sec:ZrSiO4}

The peak profiling for the data for Zircon was done using the le Bail-wizard from \textit{Jana2006}. Here, the profiling was not ideal: Due to the $K\alpha$ doublett each peak consisted of two subpeaks. Strangely, the distance between the $2\theta$ values of those two peaks did not match the calculated $2\theta$-distance between the $K\alpha$ lines. Therefore, the fitting algorithm calculated basically one pseudo-Voigt curve for each peak instead of a composition of two (see fig.~\ref{fig:demoShitPeaks}) and the wRp value never fell below 30. Anyhow, we continued refining the structure using the Rietveld-wizard of \textit{Jana2006}. The refined parameters are \par 
\centerline{\boxed{$a=b=$ \SI{6.608}{\angstrom}, \quad $c=$\SI{5.988}{\angstrom}}} \par
The result is the profile shown in fig.~\ref{fig:ZrSiO4Prof}. In this figure it can be clearly seen, that the profiling was not completely successful since the plot of the difference shows a few very high values near the peaks. This could either be due to a unskilled usage of \textit{Jana2006} or a unknown variation in the setup. In spite of this, the structure of Zircon (shown in fig.~\ref{fig:ZrSiO4Struct}) aligns to the literature~\cite{Hanchar2003}. The typical ZrO\textsubscript{8} trigondodecahedra and the SiO\textsubscript{4} tetrahedra are highlighted in the polyhedral figure style. 

\begin{figure}[ht]
    \centering
    \includegraphics[angle = 90, width = 0.7\linewidth]{Bilder/Auswertung/ZrSiO4/ZrSiO4DemoPeak.png}
    \caption{Due to the different distances between the $K\alpha$ lines and the two subpeaks of each peak, the profile could not be calculated exactly, but just as one pseudo-Voigt curve instead of a composition of two.}
    \label{fig:demoShitPeaks}  
\end{figure}

\begin{figure}[ht]
    \centering
    \includegraphics[angle = 90, width = 0.7\linewidth]{Bilder/Auswertung/ZrSiO4/ZrSiO4DataAll.png}
    \caption{For the Zircon data in red, the whole calculated profile is shown in black and the difference curve in blue.}
    \label{fig:ZrSiO4Prof}
\end{figure}


\begin{figure}[ht]
    \centering
    \includegraphics[width = \linewidth]{Bilder/Auswertung/ZrSiO4/ZrSiO4StructurePolyhedral.png}
    \caption{A model of the refinded Zircon structure is shown in a polyhedral style. Here, the typical ZrO\textsubscript{8} trigondodecahedra (in green) and the SiO\textsubscript{4} tetrahedra (in blue) can be seen.}
    \label{fig:ZrSiO4Struct}
\end{figure}

