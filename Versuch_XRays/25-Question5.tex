%Matteo Kumar - Leonard Schatt
% Fortgeschrittenes Physikalisches Praktikum
 
\section{X-ray diffraction}\label{sec:Q5}

Diffraction is the deflection of waves at an obsticle. As described in section~\ref{sec:Q2}, X-rays can be scattered coherently and incoherently by atoms. \par 
One way to describe diffraction is the theory of Laue. When X-rays fall on the lattice, constructive interferece can be seen, if the path difference $\Delta s$ fulfills 
\begin{equation}
    \Delta s = m\lambda \qquad (m \in \mathbb{N}),
\end{equation}
which can be rewritten as 
\begin{equation}
    \mathbf{k_0} - \mathbf{k} = \mathbf{G}
    \label{eq:laue}
\end{equation}
with the wave vector of the incoming beam $\mathbf{k_0}$, the wave vector of the deflected beam $\mathbf{k}$ and the reciprocal lattice vector $G$. In eq.~\ref{eq:laue} can be seen, that for a fixed incoming direction not every wavelength fulfills the conditions for constructive interference. Therefore, either a continous spectrum has to be used which leads to maxima in the intensity in certain directions or when using a fixed wavelength, the crystal has to be orientated properly~\cite{Demtroeder.2016}. \par 
In contrast, the Bragg theory uses the angle between the incoming beam and the lattice plane to describe deffraction. Constructive interference can be observed, if the Bragg equation 
\begin{equation}
    2d\sin\vartheta = m\lambda \qquad (m \in \mathbb{N})
\end{equation}
is fulfilled. $d$ describes the distance of the lattice planes and $\vartheta$ the angle mentioned above. By rotating the crystal the different lattice plane distances $d_i$ can be calculated because of the different diffraction angles $\vartheta_i$~\cite{Demtroeder.2016}.
