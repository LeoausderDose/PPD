%Matteo Kumar - Leonard Schatt
% Fortgeschrittenes Physikalisches Praktikum

% 1. Kapitel Einleitung

\chapter{Introduction}
\label{chap:einleitung}

In the following, we will present our research on X-ray diffraction, a versatile measurement method applied in various fields, including industrial applications, solid-state physics, and material science. One of the significant advantages of using X-rays is their shorter wavelength, which allows us to push the limits of resolution beyond the capabilities of optical microscopy. This enables the representation of smaller structures with higher precision. However, constructing X-ray lenses is a highly complex task due to the limited refractive index of most materials in this spectral range, which is typically close to or slightly below one. Consequently, the indirect method of "X-ray diffraction" is preferred, as it offers a means to access information at the nanoscale.

